\begin{listing}[!ht]
  \centering
\begin{minted}[fontsize=\footnotesize]{python}
import torch

class WeeNet(torch.nn.Module):
    def __init__(
        self, in_c: int, out_c: int, kdim: int, stride: int, pad: int
    ):
        super(WeeNet, self).__init__()
        self.layer1 = torch.nn.Conv2d(
            in_c,
            out_c,
            kernel_size=(kdim, kdim),
            stride=(stride, stride),
            padding=(pad, pad),
        )

    def forward(self, x: torch.Tensor) -> torch.Tensor:
        out = torch.nn.functional.relu(self.layer1(x))
        return out

model = WeeNet(*args) # initialize model
y = model(x) # run inference using input data x
\end{minted}
\caption{Simple DNN definition in PyTorch}
\label{lst:model:weenet}
\end{listing}
